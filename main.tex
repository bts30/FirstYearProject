% % % % % % % % % % % % % % % % % Preamble for notes % % % % % % % % % % % % % % % % % % % % % % % %
\documentclass[12pt,oneside,a4paper]{article}


% % % % % Sprogpakker og Layout % % % % % % % % %
\usepackage[left=2.5cm,top=2.0cm,bottom=1.5cm, right=3.0cm]{geometry}
\usepackage{ulem}
\usepackage[danish,english]{babel}
\usepackage[utf8]{inputenc}
\linespread{1.3}       % simulerer Word 1.5 line spacing

% % % % % % % % % % Øvrige vigtige pakker % % % % %
\usepackage{graphics}
\usepackage{amsmath}
\usepackage{amssymb}
\usepackage{url}
\usepackage{cancel}
\usepackage{booktabs}
\usepackage{pdfpages}
\usepackage{mathpazo}
\usepackage[section]{placeins}
\usepackage{caption}
\usepackage{wrapfig}
\usepackage{tocloft}
\usepackage{subfig}
\usepackage{fancyhdr}
%Muliggør 'flere figurer i én' Eksempel på anvendelse (indenfor figure-environment):
% \subfloat[undercaption 1]{\label{label 1}\includegraphics[bredde 1]{billede 1}}
% \subfloat[undercaption 2]{\label{label 2}\includegraphics[bredde 2]{billede 2}}
% \caption{overordnetcaption}
% \label{overordnet label}
\usepackage{lscape}
%Omdanner en del af dokumentet til landscape. Angives med \begin{landscape}
\usepackage{framed}
%Sætter en ramme om et område begrænset af \begin{framed} og \end{framed}.
%Allows footnotes
\usepackage{footnote}
\setlength{\parindent}{0cm}
\setlength{\parskip}{0.3cm}
% % % % % % COMMANDS % % % % % % % %

\begin{document}

\selectlanguage{english}
% % % % % % % % % % % % % % % % % % % % % % Forside % % % % % % % % % % % % % % % % % % % % % % % % %
\pagenumbering{roman}

\begin{center}
{\textsc {\LARGE \bf{Københavns Universitet \\[0.3cm]  Bachelorstudiet i fysik}}}\\[1.5cm]
{\textsc {\Large \bf Førsteårsprojekt 2017}}\\[0.8cm]
{\Large Projekt nummer: 2017-06}\\[1cm]

\rule{15cm}{0.01cm}\\[1cm]
{\LARGE\bf  Bouncing ball}\\ [0.5cm]
\rule{15cm}{0.01cm}\\[1cm]
\end{center}

\vfill
{\large Forfattere:}\\
{\large \hspace*{1cm} \makebox[6cm][l]{Andreas M. Faber}  \hspace{1cm} KU- ID: \makebox[2cm][l]{QZJ517} \\
{\large \hspace*{1cm} \makebox[6cm][l]{Benjamin T. Søgaard}   \hspace{1cm} KU- ID: \makebox[2cm][l]{MGX877} \\
{\large \hspace*{1cm} \makebox[6cm][l]{Joachim J. Kønigslieb}   \hspace{1cm} KU- ID: \makebox[2cm][l]{GWC666} \\

{\large Vejledere:}\\
{\large \hspace*{1cm} \makebox[6cm][l]{Jörg Helge Müller}  \hspace{1cm} Email: \makebox[2cm][l]{muller@nbi.ku.dk} \\

\vfill

{\large Rapporten omfatter {\bf 1} siders hovedtekst og {\bf 1} siders appendix.}

{\large Rapporten er indsendt som en pdf-fil den 17 marts 2017. }

\normalsize


% % % % % % % % % % % % % % % % % % % % % % Abstract  og Indholdsfortegnelse % % % % % % % % % % % % % % % % 
\newpage
\begin{abstract}
Kort resumé, gerne på både dansk og engelsk.
\end{abstract}

\newpage

\tableofcontents


% % % % % % % % % % % % % % % % % % % % % % Indhold % % % % % % % % % % % % % % % % % % % % % % % % %
\newpage
\pagenumbering{arabic}
\section{Introduction to chaos}

asdfagaffhåijpsfgh

\section{The experiment}

\section{Model of the system}
We will model our experiment as a 1-dimensional system that is we will not account for sideways movements. On figure \ref{bounces} two consecutive bounces can be seen where the later is bigger due to the plate is moving upwards at the time of impact.
\begin{figure}[h]
	\centering
	\includegraphics[width=0.6\textwidth]{Figures/bounceplot.eps}
	\caption{Plot of two bounces on the vibrating plate}
	\label{bounces}
\end{figure}
The $k$th bounce can be completely described by the time $t_k$ it left the plate and the velocity $v_k$ at that instance. Time and phases $\phi_k$ can be used interchangeably through the relation ship $\phi=2\pi f t$.

To describe the system during a specific bounce we denote the distance of the ball above the plates rest position $y_b(t)$. Similarly we define $y_p(t)$ as the displacement of the vibrating  plate from it equilibrium. For a given frequency $f$, phase shift $\phi$ and amplitude $A$ the equation of motion for the plate is
\begin{equation}
	y_p(t)= A \sin(2\pi f t+ \phi)
	\label{platey}
\end{equation}
In air the ball is essentially a body in free fall, and since it is quite small and dense, the air resistance can be ignored. Thus the equation of motion for the ball $y_b(t)$ for a specific initial condition $v_i$, $y_i$ at time $t_i$ is simply 
\begin{equation}
	y_b(t) = -\frac{g}{2}(t-t_i)^2+v_i(t-t_i)+y_i
\end{equation}
To model the impact between the ball and the plate we use a constant coefficient of restitution $C_r$ although more complex models have been suggested. For a plate at rest the velocity after the impact is proportional to the velocity before by $C_r$ such that $v_f=C_rv_i$. For a moving plate we simplify transform into the coordinate system where the plate is at rest and then transform back. Deriving equation (\ref{platey}) with respect to $t$ yields the velocity of the plate, which gives the following relationship between $v_{in}$ and $v_{out}$
\begin{equation}
	v_{out} = C_r(v_{in}-v_b)+v_b= 
\end{equation}

% % % % % % % % % % % % % % % % % % % % % % Appendix % % % % % % % % % % % % % % % % % % % % % % % % %
\newpage
\appendix
\section{Appendix}
Her er appendix. Husk at appendix ikke indgår i bedømmelsen af rapporten. Det kan bruges til supplerende udregninger, figurer eller programkode, som ikke er nødvendig for forståelsen af rapportens indhold.
\subsection{Udledning af faseforskydningen for en dæmpet, dræven simpel harmonisk oscilator.}
The equation of motion for a driven simple harmonic oscilator will be of the form:
\begin{align*}
my'' + by' + ky = F\sin(\omega t)\\
y'' + \frac{b}{m}y' + \frac{k}{m} y = F \sin(\omega t)
\end{align*}
We will then write the generel solution as a sum of the particular and the homogenues solutions:
\begin{align*}
y_g = y_{h} + y_p
\end{align*}
The homogenues solution for the damped harmonic oscilater will go to zero with time. We claim it is only the particular solution that matters. To solve this, we will guess the solution to be of the form:
\begin{align*}
y_p = A \sin (\omega t) + B \cos(\omega t)
\end{align*}
We will perform the differetiations and plug into the equation to get:
\begin{align*}
&m\left(-A\omega^2\sin(\omega t) - B\omega^2\cos(\omega t)\right) + b\left(A\omega\cos(\omega t) - B\omega\sin(\omega t)\right) + k\left( A\sin(\omega t) + B\cos(\omega t) \right)\\
&=\sin(\omega t) \left( -mA\omega^2 -bB\omega + kA \right) + \cos(\omega t)\left(-mB\omega^2 + bA\omega + Bk\right)
\end{align*}
After collecting the coefficents, we can match the left side, with the right side:
\begin{align}
F &= -mA\omega^2 -bB\omega + Ak \\ \label{lignA}
0 &= -mB\omega^2 + bA\omega + Bk \label{lignB}
\end{align}
We isolate an expression for $B$ in (\ref{lignB})
\begin{align*}
B &= \frac{-bA\omega}{-m\omega^2+k}
\end{align*}
Which we will proceed to plug into (\ref{lignA}), to obtain an expression for $A$:
\begin{align*}
A = \frac{F}{-m\omega^2+k+\frac{b^2\omega^2}{-m\omega^2+k}} = \frac{F\left(k-m\omega^2\right)}{\left(k-m\omega^2\right)^2+b^2\omega^2}
\end{align*} 
We can write $A$ and $B$ independetly of eachother:
\begin{align*}
B &= \frac{-bF\omega}{\left(k-m\omega^2\right)^2+b^2\omega^2}\\
A &=  \frac{F\left(k-m\omega^2\right)}{\left(k-m\omega^2\right)^2+b^2\omega^2}
\end{align*}
We would now like to express $y_p$ in the shape:
\begin{align*}
y_p = L \sin(\omega t + \phi)
\end{align*}
If we write the sine addition identity,
\begin{align*}
L sin(\omega t + \phi) = L\cos(\phi)\sin(\omega t) + L\sin(\phi)\cos(\omega t)
\end{align*}
We see, that our amplitude must obey,
\begin{align*}
A = L\cos(\phi)\\
B = L\sin(\phi)
\end{align*}
if we are to get our initial guess of
\begin{align*}
y_p = A \sin (\omega t) + B \cos(\omega t)
\end{align*}
The phaseshift will then be:
\begin{align*}
\frac{B}{A} &= \frac{L\sin(\phi)}{L\cos(\phi)} = \tan(\phi)\\
\phi &= \arctan\left(\frac{B}{A}\right)\\
\phi &= \arctan\left( \frac{-bF\omega}{\left(k-m\omega^2\right)^2+b^2\omega^2} \cdot \frac{\left(k-m\omega^2\right)^2+b^2\omega^2}{F\left(k-m\omega^2\right)} \right)\\
\phi &= \arctan\left( \frac{b\omega}{m\omega^2-k} \right)
\end{align*}
\end{document}
